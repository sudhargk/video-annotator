\acknowledgements
I would like to to express my sincere thanks and deep sense of indebtedness to my guide Prof. Hema A. Murthy for her guidance and motivation throughout my work. Her inspiring suggestions pondered me to solve problems efficiently. I am grateful to my guide also for providing DONLAB servers: Dell-d1 and Dell-d2  without which none of my experiments could be done.

\par I would like to thank my co-partner Abil N George for his support and contribution in the toolkit development. Also my deepest gratitudes to Prof. C. Chandra Sekhar and Anil Kumar Chilli for allowing us to use Systems with GPU in Speech and Vision Laboratory during the development of the toolkit. 

\par Also I like to acknowledge all my colleagues at DONLAB(IIT Madras) who helped me throughout my research. 
 
\par Lastly, I am thankful to my parents for all the moral support and the amazing chances they have given me over the years.

%%%%%%%%%%%%%%%%%%%%%%%%%%%%%%%%%%%%%%%%%%%%%%%%%%%%%%%%%%%%%%%%%%%%%%
